%!TeX encoding = utf8
%!TeX spellcheck = it_IT
%!TeX source = ../ML.tex

\def\ml{Machine Learning}

\lecture{26/9}

\subsection{Contestualizzazione}

L'obiettivo del \ml\ è insegnare ad un sistema un compito preciso, costruendo un modello utilizzabile per predire il corretto output, dopo aver esaminato un gran numero di esempi. Tale metodo di apprendimento è detto \emph{per generalizzazione}.

È utile per esempio quando l'approccio teorico a un determinato problema è difficilmente praticabile, o quando i dati in input sono poco accurati, affetti da errore o incompleti.

\begin{definition}[Machine Learning]
Il \emph{\ml} studia e propone metodi per inferire funzioni o correlazioni che, a partire da dati osservati, producano il corretto output sui \emph{samples} forniti e li generalizzino con ragionevole accuratezza.
\end{definition}

\subsection{Terminologia}

Un \emph{machine learning system} si compone di \emph{dati}, \emph{tasks}, \emph{modelli}, \emph{algoritmi di apprendimento} e \emph{validazione}.

\begin{definition}[Dati]
I \emph{dati} rappresentano le \emph{esperienze disponibili}. Possono essere organizzati in un certo numero $l$ di istanze $x_p$ (\emph{samples}, \emph{instances}), ciascuno contenente $n$ attributi (\emph{features}). Con $x_{p,j}$ indicheremo l'attributo $j$-esimo della $p$-esima istanza. 
\end{definition}

%Risulta conveniente indicare i valori dei vari attributi come vettori di dimensione $k$ (dove $k$ è il numero dei valori possibili) con componenti tutte nulle a parte una, in modo che valori diversi siano ortogonali.

\begin{remark}
	Se un attributo può assumere un numero finito di valori, risulta spesso conveniente rappresentarlo come un vettore di dimensione $k$ (dove $k$ è il numero dei valori possibili) con componenti tutte nulle a parte una.
\end{remark}

\begin{example}
Se i valori possibili sono i tre colori \emph{rosso} ($R$), \emph{verde} ($G$), \emph{blu} ($B$), si pone
\begin{equation}
R=(1,0,0),\quad G=(0,1,0),\quad B=(0,0,1).
\end{equation}
\end{example}

\begin{definition}[Rumore, outliers]
Il \emph{rumore} è l'aggiunta di fattori esterni dovuta al processo di misura (e non alla legge soggiacente). Gli \emph{outliers} sono dati che si collocano molto lontani rispetto agli altri.
\end{definition}

I \emph{tasks} sono in genere di due tipologie (ma ce ne sono altre):

\begin{definition}[Supervised learning]
Nel \emph{supervised learning} sono dati dei \emph{samples} di una funzione $f$ ignota, nella forma \[\langle\text{\emph{input}},\,\text{\emph{output}}\rangle\] si tratta di trovare una buona approssimazione di $f$. Gli input sono detti anche \emph{variabili indipendenti}, gli ouput \emph{variabili dipendenti} o \emph{risposte}. Se $f$ è a valori discreti, il problema si dice di \emph{classificazione}. Se l'output è costituito da valori reali, allora si parla di \emph{regressione}.
\end{definition}

\begin{definition}[Unsupervised learning]
Nell'\emph{unsupervised learning} non si dispone di input e output nelle istanze, ma solo di dati non etichettati. Un problema tipico è quello di raggruppare tali dati secondo determinati criteri.
\end{definition}

Noi ci occuperemo soprattutto di supervised learning.

\begin{definition}[Modello, ipotesi]
Il modello cerca di descrivere la relazione tra i dati con un \emph{linguaggio}, legato alla rappresentazione dei dati. Le \emph{ipotesi} sono le funzioni $h_w$ proposte dal modello per approssimare la “vera” funzione $f$. Le ipotesi sono indicizzate da parametri ($w$) e formano uno \emph{spazio delle ipotesi} $H$.
\end{definition}

In generale non esiste un modello \emph{ottimo}: se un modello di apprendimento è il migliore in qualche problema, sarà peggiore di altri in altri problemi. Questo concetto è noto come \emph{No Free Lunch Theorem}, ovvero “non c'è un pranzo gratis” (mah, sarà qualche detto inglese, ndr). In ogni caso, questo non significa che tutti i modelli siano equivalenti.

\begin{definition}[Algoritmo di apprendimento]
Un \emph{algoritmo} di apprendimento si occupa di cercare nello spazio delle ipotesi $H$ (di un modello fissato) la migliore approssimazione della funzione $f$.
\end{definition}

Come definiamo \emph{buona approssimazione}? Si utilizza una \emph{loss function} $L\big(h(x),d\big)$, che misura la distanza tra $h(x)$ e $d$, dove $d=f(x)$ è il valore osservato.

\begin{definition}[Errore]
L'errore è definito da
\begin{equation}
E=\frac 1l\sum_{i=1}^{l}L\big(h(x_i),d_i\big)
\end{equation}
dove $x_i$ sono le istanze e $d_i=f(x_i)$ i valori osservati.
\end{definition}

\begin{example}
Nei problemi di regressione spesso come loss function si usa \begin{equation}
L\big(h(x),d\big)=\big(d-h(x)\big)^2
\end{equation}
e in tal caso l'errore si dice \emph{errore quadratico medio} (MSE).

Se siamo di fronte a un problema di classificazione, allora è più conveniente usare la loss function che vale $1$ se i suoi due argomenti sono uguali (e dunque la classificazione è corretta) e $0$ altrimenti.
\end{example}

\begin{remark}
In \ml, quando si parla di \emph{performance}, si fa riferimento all'accuratezza predittiva, non all'efficienza computazionale.
\end{remark}