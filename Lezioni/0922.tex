%!TeX encoding = utf8
%!TeX spellcheck = it_IT
%!TeX source = ../ML.tex

\lecture{22/9}

\subsection{Alcune indicazioni}

Un paio di cose:
\begin{itemize}
\item Per creare paragrafi numerati, non va usato \lstinline|\section|, ma \lstinline|\subsection|;
\item Non usate il grassetto, ma il \emph{corsivo};
\item Usate \texttt{equation}  per le equazioni, in modo che vengano tutte numerate.
\end{itemize}
\begin{equation}
e^{z}=\sum_{n=0}^{+\infty} \frac{z^n}{n!}.
\end{equation}

\lipsum[3]

\begin{lstlisting}[language=C]
#include <stdio.h>
int main()
{
   // printf() displays the string inside quotation
   printf("Hello, World!");
   return 0;
}
\end{lstlisting}