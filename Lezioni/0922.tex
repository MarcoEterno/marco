%!TeX encoding = utf8
%!TeX spellcheck = it_IT
%!TeX source = ../ML.tex

\lecture{22/9}

\subsection{Alcune indicazioni}

Un paio di proposte, per uniformità:
\begin{itemize}
\item Ogni lezione inizia con \lstinline|\lecture{|$\langle$\textit{data}$\rangle$\lstinline|}|, dove la data è del tipo \texttt{29/2}.
\item Per creare paragrafi numerati, non va usato \lstinline|\section|, ma \lstinline|\subsection|;
\item Non usate il grassetto, ma il \emph{corsivo}, soprattutto quando s'introduce un \emph{nuovo termine}, e poi si può usare il nuovo termine anche senza corsivo;
\item Usate \texttt{equation}  per le equazioni, in modo che vengano tutte numerate; le equazioni a fine frase terminano con il punto, ma magari evitiamo altro tipo di punteggiatura alla fine di equazioni.
\item usare molto gli ambienti \lstinline|definition|, \lstinline|theorem|, \lstinline|example|, \lstinline|remark|\dots
\item negli elenchi, il “;” alla fine degli \emph{item} e il punto alla fine dell'ultimo. Ma se sono frasi lunghe va bene anche il punto dappertutto.
\end{itemize}

\begin{definition}[Tensore]
Un \emph{tensore} è ciò che ruota come un tensore.
\end{definition}

\begin{equation}
e^{z}=\sum_{n=0}^{+\infty} \frac{z^n}{n!}.
\end{equation}

\begin{lstlisting}[language=C]
#include <stdio.h>
int main()
{
   // printf() displays the string inside quotation
   printf("Hello, World!");
   return 0;
}
\end{lstlisting}