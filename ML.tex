%!TeX encoding = utf8
%!TeX spellcheck = it_IT

\documentclass[a4paper, 11pt]{book}

\usepackage[T1]{fontenc}
\usepackage[utf8]{inputenc}
\usepackage[italian]{babel}
\usepackage{amsmath, amssymb, amsthm}
\usepackage{geometry, emptypage}
\usepackage{graphicx, subfig}
\usepackage{enumitem, lipsum}
\usepackage{marginnote, xspace}
%\usepackage[autostyle,italian=guillemets]{csquotes}
%\usepackage[style=philosophy-modern, backend=biber, square, hyperref, backref]{biblatex}
\usepackage{listings}
\usepackage[eulerchapternumbers, palatino=false, parts=false]{classicthesis}
\usepackage{arsclassica}

% Font
\IfFileExists{MinionPro.sty}{
	\usepackage[noopticals, mathlf, minionint, loosequotes]{MinionPro}
	\usepackage[toc, enum]{tabfigures}
	\usepackage{relsize}
	\let\oldsum\sum\let\oldprod\prod
	\def\sum{\mathlarger{\oldsum}}%
	\def\prod{\mathlarger{\oldprod}}
}{
	\usepackage[proportional, oldstyle]{cochineal}
	\usepackage[cochineal, vvarbb]{newtxmath}
}

% Figures
\graphicspath{{Figure/}}
\captionsetup{format=hang}

% Layout
\geometry{%showframe,
	width=30pc, height=54pc, vmarginratio=1000:1414}
\setlength\parindent{1em}
\newcommand\setupheading[1]{%
	\cleardoublepage\phantomsection
	\markboth{\spacedlowsmallcaps{#1}}{\spacedlowsmallcaps{#1}}%
}

% Bibliography
%\addbibresource{Bibliography.bib}
%\renewcommand*{\nameyeardelim}{\addcomma\space}

% Headings
\renewcommand{\sectionmark}[1]{
	\markright{\thesection\enspace\textit{#1}}
} 
\clearplainofpairofpagestyles

% Code
\lstset{language=C,
 	basicstyle=\ttfamily\small,
	keywordstyle=\color{RoyalBlue},
	commentstyle=\color{black!20!Emerald}\footnotesize\itshape,
	columns=flexible,
	aboveskip=\bigskipamount,
	stringstyle=\color{Orange},
	numbers=none,
	tabsize=4,
	showstringspaces=false,
	breaklines=true,
	frameround=ftff,
	frame=lines,
	backgroundcolor=\color{white},%\color[gray]{.99},
	escapeinside={£}{£},
	extendedchars=true,
	literate={à}{{\`a}}1 {è}{{\`e}}1 {ù}{{\`u}}1 {ö}{{\"o}}1
}

% Theorems
\newtheoremstyle{plain}{}{}{\itshape}{}{}{}{ }{%
	\textbf{\thmname{#1}\thmnumber{ #2.}}\thmnote{ (#3).}}
\newtheoremstyle{definition}{}{}{}{}{}{}{ }{%
	\textbf{\thmname{#1}\thmnumber{ #2.}}\thmnote{ (#3).}}

\theoremstyle{plain}
\newtheorem{theorem}{Theorema}[chapter]
\newtheorem{corollary}[theorem]{Corollario}
\newtheorem{proposition}[theorem]{Proposizione}
\newtheorem{lemma}[theorem]{Lemma}

\theoremstyle{definition}
\newtheorem{definition}[theorem]{Definizione}
\newtheorem{remark}[theorem]{Osservazione}
\newtheorem{example}[theorem]{Esempio}

\numberwithin{equation}{chapter}
\numberwithin{figure}{chapter}
\numberwithin{table}{chapter}

% Other
\hypersetup{%hidelinks,
	linkcolor=RoyalBlue}
\microtypesetup{kerning=true}
\let\marginpar\marginnote
\setlist{itemsep=0pt}
\def\ml{Machine Learning\xspace}

% Dates
\usepackage{Files/datemath}
\newdate{lecture}
\setdaynames{domenica}{luned\`i}{marted\`i}{mercoled\`i}{gioved\`i}{venerd\`i}{sabato}
\setmonthnames{gennaio}{febbraio}{marzo}{aprile}{maggio}{giugno}{luglio}
	{agosto}{settembre}{ottobre}{novembre}{dicembre}
\newcommand\lecture[1]{%
	\setdate{lecture}{#1/2019}%
	\calcdayofweek{lecture}%
	\section{Lezione di \dayofweek{lecture} \theday{lecture} \monthname{lecture}}%
}

\begin{document}

\mainmatter

% !TeX source = ../Tesi.tex

%****************************************************************
% Title Page
%****************************************************************

\begin{titlepage}
\newgeometry{top=1in, bottom=1in, left=1.5in, right=1.5in}
\begin{center}
\includegraphics[width=3cm]{LogoUnipi}\par\bigskip
%\includegraphics[width=5cm]{ScrittaUnipi}\par\bigskip
{\large\spacedallcaps{Universit\`a di Pisa}}\\[1ex]
{\large Dipartimento di Informatica}

\vspace{\stretch{1}}

{\Large\spacedallcaps{Machine Learning}}\par\bigskip
{\large\itshape Appunti dalle lezioni del Prof. Alessio Micheli}

\vspace{\stretch{1}}

\begin{center}
\itshape
\begin{tabular}{r@{\hspace{.5em}}l}
Pasquale & Miglionico\\[1.5ex]
Guido & Narduzzi\\[1.5ex]
Enrico & Negri\\[1.5ex]
Filippo & Quattrocchi\\[1.5ex]
Francesco & Zigliotto
\end{tabular}
\end{center}

\vspace{\stretch{1}}

\spacedlowsmallcaps{Anno accademico 2019--2020}
\end{center}
\restoregeometry
\end{titlepage}

\input{Files/Indice}

\chapter{Introduzione}

%!TeX encoding = utf8
%!TeX spellcheck = it_IT
%!TeX source = ../ML.tex

\makeatletter
\def\ml{Machine Learning}
\makeatother

\lecture{26/9}

%\subsection{Contestualizzazione}

L'obiettivo è insegnare ad un sistema un compito preciso, costruendo un modello utilizzabile per predire il corretto output, dopo aver esaminato un gran numero di esempi. Tale metodo di apprendimento è detto \emph{per generalizzazione}.

È utile per esempio quando l'approccio teorico a un determinato problema è difficilmente praticabile, o quando i dati in input sono poco accurati, affetti da errore o incompleti.

\subsection{Terminologia}

\begin{definition}[Machine Learning]
Il \emph{\ml} studia e propone metodi per inferire funzioni o correlazioni che, a partire da dati osservati, producano il corretto output sui \emph{samples} forniti e li generalizzino con ragionevole accuratezza.
\end{definition}

Un \emph{machine learning system} si compone di \emph{dati}, \emph{tasks}, \emph{modelli}, \emph{algoritmi di apprendimento} e \emph{validazione}.

\begin{definition}[Dati]
I \emph{dati} rappresentano le \emph{esperienze disponibili}. Possono essere organizzati in un certo numero $l$ di istanze $x_p$ (\emph{samples}, \emph{instances}), ciascuno contenente $n$ attributi (\emph{features}). Con $x_{p,j}$ indicheremo l'attributo $j$-esimo della $p$-esima istanza. 
\end{definition}

Risulta conveniente indicare i valori dei vari attributi come vettori di dimensione $k$ (dove $k$ è il numero dei valori possibili) con componenti tutte nulle a parte una, in modo che valori diversi siano ortogonali.
\begin{example}
Se i valori possibili sono i tre colori \emph{rosso} ($R$), \emph{verde} ($G$), \emph{blu} ($B$), si pone
\begin{equation}
R=(1,0,0),\quad G=(0,1,0),\quad B=(0,0,1).
\end{equation}
\end{example}

\begin{definition}[Rumore, outliers]
Il \emph{rumore} è l'aggiunta di fattori esterni dovuta al processo di misura (e non alla legge soggiacente). Gli \emph{outliers} sono dati che si collocano molto al di fuori, rispetto agli altri.
\end{definition}

I \emph{tasks} sono in genere di due tipologie (ma ce ne sono altre):

\begin{definition}[Supervised learning]
Nel \emph{supervised learning} sono dati dei \emph{samples} di una funzione $f$ ignota, nella forma \[\langle\text{\emph{input}},\,\text{\emph{output}}\rangle\] si tratta di trovare una buona approssimazione di $f$. Gli input sono detti anche \emph{variabili indipendenti}, gli ouput \emph{variabili dipendenti} o \emph{risposte}. Se $f$ è a valori discreti, il problema si dice di \emph{classificazione}. Se l'output è costituito da valori reali, allora si parla di \emph{regressione}.
\end{definition}

\begin{definition}[Unsupervised learning]
Nell'\emph{unsupervised learning} non si dispone di input e output nelle istanze, ma solo di dati non etichettati. Un problema tipico è quello di raggruppare tali dati secondo determinati criteri.
\end{definition}

Noi ci occuperemo soprattutto di supervised learning.

\begin{definition}[Modello, ipotesi]
Il modello cerca di descrivere la relazione tra i dati con un \emph{linguaggio}, legato alla rappresentazione dei dati. Le \emph{ipotesi} sono le funzioni $h_w$ proposte dal modello per approssimare la “vera” funzione $f$. Le ipotesi sono indicizzate da parametri ($w$) e formano uno \emph{spazio delle ipotesi} $H$.
\end{definition}

In generale non esiste un modello \emph{ottimo}: se un modello di apprendimento è il migliore in qualche problema, sarà peggiore di altri in altri problemi. Questo concetto è noto come \emph{No Free Lunch Theorem}, ovvero “non c'è un pranzo gratis” (mah, sarà qualche detto inglese, ndr). In ogni caso, questo non significa che tutti i modelli siano equivalenti.

\begin{definition}[Algoritmo di apprendimento]
Un \emph{algoritmo} di apprendimento si occupa di cercare nello spazio delle ipotesi $H$ (di un modello fissato) la migliore approssimazione della funzione $f$.
\end{definition}

Come definiamo \emph{buona approssimazione}? Si utilizza una \emph{loss function} $L(h(x),d)$, che misura la distanza tra $h(x)$ e $d$, dove $d=f(x)$ è il valore osservato.

\begin{definition}[Errore]
L'errore è definito da
\begin{equation}
E=\frac 1l\sum_{i=1}^{l}L\big(h(x_i),d_i\big),
\end{equation}
dove $x_i$ sono le istanze e $d_i=f(x_i)$ i valori osservati.
\end{definition}

\begin{example}
Nei problemi di regressione spesso si usa \[L(h(x),d)=\big(d-h(x)\big)^2\]e in tal caso l'errore si dice \emph{errore quadratico medio} (MSE).

Se siamo di fronte a un problema di classificazione, allora è più conveniente usare la \emph{loss function} che vale $1$ se i suoi due argomenti sono uguali (e dunque la classificazione è corretta) e $0$ altrimenti.
\end{example}

\begin{remark}
In \ml, quando si parla di \emph{performance}, si fa riferimento all'accuratezza predittiva, non all'efficienza computazionale.
\end{remark}
%!TeX encoding = utf8
%!TeX spellcheck = it_IT
%!TeX source = ../ML.tex

\lecture{27/9}

\subsection{Fitting e overfitting}

\begin{definition}[Validazione]
La \emph{validazione} valuta la capacità di generalizzazione di una determinata ipotesi, misurandone l'accuratezza.
\end{definition}

%Chiaramente il problema del \ml è che è mal condizionato: la soluzione non è unica e non è chiaro quale sia la migliore.

In generale non possiamo assumere che se un'ipotesi $h$ approssima bene la funzione $f$ sui samples di allenamento, allora $h$ approssima $f$ anche su nuove istanze. C'è per esempio il problema dell'\emph{overfitting}:
\begin{definition}[Overfitting]
L'\emph{overfitting} avviene quando sottostimiamo l'errore sperimentale nel fitting e quindi aumenta il \emph{vero} errore sui dati sconosciuti. L'overfitting avviene quindi se il modello è troppo complesso e quindi è in grado di \emph{fittare il rumore}.
\end{definition}

\begin{example}[Fitting polinomiale]
Supponiamo di avere una funzione reale e di voler risolvere il problema di regressione con un \emph{fit polinomiale}. Le ipotesi sono della forma \begin{equation}
y(x,w)=\sum_{i=0}^M w_ix^i
\end{equation}
dove $w$ è il vettore dei coefficienti $w_i$. Si cerca di minimizzare l'errore quadratico medio. Si osserva che se il polinomio ha grado troppo basso i \emph{training samples} non vengono fittati correttamente (\emph{underfitting}). Aumentando troppo il grado del polinomio nel modello, l'errore sui \emph{training samples} diminuisce, mentre quello sui \emph{test samples} aumenta (e i coefficienti del polinomio aumentano di molto in modulo): questo comportamento è tipico dell'\emph{overfitting}.

Infine, a parità di grado, si nota che aumentando il numero di dati il fitting migliora molto, indipendentemente dal fatto che ci sia molto rumore.
\end{example}

\subsection{Setting semplificato}

Formalmente, quindi, disponiamo di
\begin{itemize}
\item una funzione $f$ ignota da approssimare;
\item un modello con un relativo spazio di ipotesi $H$;
\item una \emph{loss function} $L$;
\item una distribuzione di probabilità $P(x,d)$ per lo spazio dei dati, dove $d(x)$ corrisponde alla distribuzione dei dati sperimentali ad $x$ fissato ($d$ corrisponde al valore di $f(x)$ misurato sperimentalmente, quindi affetto da rumore, e per lo stesso $x$ naturalmente si possono avere più misure).
\end{itemize}
L'obiettivo teorico sarebbe minimizzare il \emph{rischio}, cioè il \emph{vero errore su tutti i dati}:
\begin{equation}
R=\int L(h_w(x),d)\,dP(x,d).
\end{equation}

È tuttavia più praticabile lavorare con il \emph{rischio empirico}, cioè l'errore sui \emph{training samples}:
\begin{equation}
R\ped{emp}=\frac{1}{l}\sum L(d-h_w(x),d).
\end{equation}
Chiaramente non bisogna minimizzare il rischio empirico, per via dell'overfitting. Occorre considerare insieme il rischio empirico e la complessitò.

Si può mostrare che, con probabilità $1-\delta$, vale
\begin{equation}
R\le R\ped{emp} + \varepsilon(1/l, VC, 1/\delta)
\end{equation}
dove $l$ è il numero di samples, $VC$ è la “complessità del modello” (il \emph{grado} del polinomio, nell'esempio del fit polinomiale) ed $\varepsilon$ è un'opportuna funzione: solitamente si assume che $\varepsilon$ sia direttamente proporzionale ai suoi due primi argomenti. Infatti, per grandi valori di $l$, il rischio $R$ è minore. Mentre per grande complessità del modello il rischio empirico $R\ped{emp}$ è minore ma il rischio $R$ può aumentare, per via dell'overfitting.

\subsection{Validazione}

La validazione è composta da due fasi:
\begin{itemize}
\item la selezione del modello (\emph{model selection}) si occupa di scegliere il modello più adatto a trattare il problema;
\item il giudizio del modello (\emph{model assessment}) valuta la capacità predittiva del modello su \emph{test samples}.
\end{itemize}
In particolare si divide l'insieme di dati in TR (\emph{training set}), VL (\emph{validation set}) and TS (\emph{test set}). A questo punto avviene la \emph{model selection}, in cui
\begin{itemize}
\item si fissa un determinato modello e si usano i dati in TR per trovare la funzione $h$ che minimizzi il rischio empirico;
\item si usano i dati in VL per determinare la bontà del modello;
\item si cicla sui due step sopra al fine di determinare il modello migliore.
\end{itemize}
Una volta ottenuto il modello migliore (già pronto per l'utilizzo, con tutti parametri fissati dai dati in TR), si effettua il \emph{model assessment} e si valuta il comportamento sui nuovi dati presenti nel TS.

\begin{example}[$k$-fold cross-validation]
Per esempio, si può dividere l'insieme di dati per la \emph{model selection} in $k$ sottoinsiemi $D_1,\dots,D_k$ e poi, per ogni $j=1,\dots,k$ utilizzare $D_j$ come VL e tutti gli altri $D_i$ come TR.
\end{example}

\subsection{Matrice di confusione}

[\dots]


\chapter{Esempi}

\lipsum[2]

%!TeX encoding = utf8
%!TeX spellcheck = it_IT
%!TeX source = ../ML.tex

\lecture{22/9}

\subsection{Alcune indicazioni}

Un paio di proposte, per uniformità:
\begin{itemize}
\item Ogni lezione inizia con \lstinline|\lecture{|$\langle$\textit{data}$\rangle$\lstinline|}|, dove la data è del tipo \texttt{29/2}.
\item Per creare paragrafi numerati, non va usato \lstinline|\section|, ma \lstinline|\subsection|;
\item Non usate il grassetto, ma il \emph{corsivo}, soprattutto quando s'introduce un \emph{nuovo termine}, e poi si può usare il nuovo termine anche senza corsivo;
\item Usate \texttt{equation}  per le equazioni, in modo che vengano tutte numerate; le equazioni a fine frase terminano con il punto, ma magari evitiamo altro tipo di punteggiatura alla fine di equazioni.
\item usare molto gli ambienti \lstinline|definition|, \lstinline|theorem|, \lstinline|example|, \lstinline|remark|\dots
\item negli elenchi, il “;” alla fine degli \emph{item} e il punto alla fine dell'ultimo. Ma se sono frasi lunghe va bene anche il punto dappertutto.
\end{itemize}

\begin{definition}[Tensore]
Un \emph{tensore} è ciò che ruota come un tensore.
\end{definition}

\begin{equation}
e^{z}=\sum_{n=0}^{+\infty} \frac{z^n}{n!}.
\end{equation}

\begin{lstlisting}[language=C]
#include <stdio.h>
int main()
{
   // printf() displays the string inside quotation
   printf("Hello, World!");
   return 0;
}
\end{lstlisting}

\end{document}